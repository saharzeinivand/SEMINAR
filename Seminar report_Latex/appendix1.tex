\chapter{جبر ریس}
%%%%%%%%%%%%%%%%%%%%%%%%%%%%%%%%%%%%%%%%
% به دستورات زیر دست نزنید
\thispagestyle{empty}
\newpage
%%%%%%%%%%%%%%%%%%%%%%%%%%%%%%%%%%%%%%%%%%
\section{جبر ریس و عدد نظم}
\begin{definition}\label{bigraded-def}
فرض کنید $R$ یک حلقه باشد و $\{R_{(i,j)}\}_{(i,j)\in {\Bbb Z^2}}$ خانواده ای از زیرگروه‌های جمعی $R$ باشد.
$R$
را یک حلقه
مدرجِ دوگانه\footnoteC{Bigraded ring}
گوئیم، هرگاه
\begin{enumerate}
\item
$R=\underset{(i,j)\in \Bbb Z^2}{\oplus } R_{(i,j)}$.
\item
برای هر $(i,j),(i',j') \in\Bbb Z^2$ داشته باشیم~~ $R_{(i,j)}R_{(i',j')}\subseteq R_{(i+i',j+j')}$.
\end{enumerate}
همچنین $R$-مدول $M$ را مدرج دوگانه نامیم هرگاه به صورت
$M=\underset{(i,j)\in \Bbb Z^2}{\oplus } M_{(i,j)}$
باشد که در آن هر
$M_{(i,j)}$
یک زیرگروه جمعی $M$ است و برای هر 
$(i,j),(i',j') \in\Bbb Z^2$،~~
$R_{(i,j)}M_{(i',j')}\subseteq M_{(i+i',j+j')}$
{\large{توجه:}}
به همین ترتیب می توان حلقه و مدول مدرج چندگانه
\footnoteC{Multigraded} 
را تعریف نمود.
\end{definition}
\begin{remark}\label{rees-reg}
فرض کنید $S=K[x_1,\ldots,x_n]$ حلقه‌ی چندجمله‌ای‌ روی میدان $K$ و $I\subseteq S$ یک ایده‌ال با تولید یکسان از درجه $d$ با مولد‌های $f_1,\ldots,f_m$ باشد. در این صورت جبر ریسِ حاصل از صافی $I$-‌اَدیک، یعنی 
$$\mathcal{R}(I)=\underset{j\in \Bbb N_0}{\oplus}I^j t^j=S[f_1t,\ldots,f_mt]=K[x_1,\ldots,x_n,f_1t,\ldots,f_mt]\subseteq S[t]$$
را می‌توان با تعریف $\degi(x_i)=(1,0)$ برای هر 
$1\leq i\leq n$
و $\degi(f_jt)=(0,1)$ برای 
$j=1,\ldots,m$،
به یک مدول مدرج دوگانه تبدیل نمود.\\
در تعریف \ref{T-bigraded}،
دیدیم که $T=S[y_1,\ldots,y_n]$ به گونه‌ای طبیعی یک حلقه مدرج دوگانه می‌باشد.
همریختی طبیعی
$\varphi:~T\longrightarrow \mathcal{R}(I)$
با ضابطه $\varphi(x_i)=x_i$ و $\varphi(y_j)=f_jt$، برای $i=1,\ldots,x_n$ و $j=1,\ldots,m$، یک همریختی پوشا بین $K$-‌جبرهای مدرج دوگانه تعریف می‌کند
که به وضوح $\mathcal{R}(I)$
را به یک $T$-‌مدول متناهی مولد مدرجِ دوگانه تبدیل می‌کند.\\
حال گیریم 
\begin{displaymath}
\xymatrix{
{\bf F}:~~0\ar[r] & {F_s}\ar[r]&{\cdots} \ar[r] & F_1 \ar[r] &F_0 \ar[r] &\mathcal{R}(I) \ar[r] &0}
\end{displaymath}
تحلیل آزاد مدرج دوگانه مینیمال $T$-‌مدول $\mathcal{R}(I)$ باشد.
که در آن برای هر
$i=0,\ldots,s$
داریم
$F_i=\underset{j}{\oplus}T(-a_{ij},-b_{ij})$.
ساختار $T$-مدولی  
$\mathcal{R}(I)$
توسط تابع $\varphi$ مشخص می‌شود و داریم
$$\mathcal{R}(I)\cong T/{\kernn\varphi}$$
بنا بر تعریف
\ref{reg-bigraded}،
 عدد نظم وابسته به $x$ برای $\mathcal{R}(I)$ به صورت زیر است
$$\reg_x(\mathcal{R}(I))=\max\left\{a_{ij}-i:~i,j\in N_0\right\}.$$
در فصل \ref{regularity-asli} نشان می‌دهیم که برای همه‌ی ایده‌ال‌های با تولید یکسان از درجه $d$،
$$\reg(I^n)\leq nd+\reg_x(\mathcal{R}(I)).$$
که در آن $\mathcal{R}(I)$
جبر ریسِ حاصل از صافی $I$-‌اَدیک، است. از این فرمول که رومر
\footnoteC{R\"omer}
در 
\cite{romer}
اثبات نمود، می‌توان نتیجه گرفت که همه‌ی توان‌های $I$ تحلیل خطی دارند هرگاه 
$\reg_x(\mathcal{R}(I))=0$.
و می‌گوییم $I$ در شرط $x$ صدق می‌کند هرگاه 
$\reg_x(\mathcal{R}(I))=0$.\\
\end{remark}