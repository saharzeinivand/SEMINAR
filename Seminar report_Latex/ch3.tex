\chapter{تحلیل آزاد مدرج و عدد نظم}\label{ch-free}
\
%%%%%%%%%%%%%%%%%%%%%%%%%%%%%%%%%%%%%%%%
% به دستورات زیر دست نزنید
\thispagestyle{empty}
\newpage
%%%%%%%%%%%%%%%%%%%%%%%%%%%%%%%%%%%%%%%%%%
با توجه به این که  ابزار کار ما در این پایان نامه حلقه‌چندجمله‌ای‌  است بیشتر قضایا و لم‌ها روی این حلقه بیان شده است.
\section{{تحلیل آزاد مدرج و اعداد بتی مدرج}}
\begin{def-nok}
$-R$
مدول $F$ را 		
{آزاد مدرج}
 گوییم، هرگاه $F$ دارای پایه‌ای باشد که همه‌ی اعضای آن همگن باشد به عبارتی خانواده‌ای از اعداد صحیحِ $\{n_i\}_{i\in I}$ چنان باشد که $F\cong \underset{i\in I}{\oplus}R(n_i)$ \\
 بنابراین می‌توان به سادگی نشان داد که هر مدول مدرج، تصویر همریختِ یک مدول آزادِ مدرج است، یعنی اگر $M$ یک $-R$مدول مدرج باشد و $\{m_i\}_{i\in I}$ مجموعه مولدی برای $M$ باشد که برای هر
 $i\in I$،
 $\degi(m_i)=n_i$
 آنگاه $-R$همریختی زیر همگن  و پوشاست
\begin{align*}
 \varphi:& \underset{i\in I}{\oplus} R(-n_i)\longrightarrow M\\
&e_i \longmapsto m_i
 \end{align*}
\end{def-nok}
\begin{definition}
فرض کنید $M$ یک $R$‌-‌مدول مدرج باشد، آنگاه یک تحلیل آزاد مدرج
 از $M$،‌ رشته‌ای دقیق به صورت 
\begin{displaymath}
\xymatrix{
{\cdots} \ar[r] & {F_i}\ar[r]^{d_{i}}&{F_{i-1}} \ar[r]^{d_{i-1}}  &{\cdots} \ar[r]^{d_2} & F_1 \ar[r]^{d_1} &F_0 \ar[r]^{d_0} &M \ar[r] &0}
\end{displaymath}
است که در آن $F_i$ها، $R$‌-‌مدول‌های آزاد مدرج هستند و $d_i$ها، $R$‌-‌همریختی‌های همگن‌اند.
\end{definition}
\begin{definition}\label{mini}
فرض کنید $(R,{\mm})$ یک حلقه ${}^*$موضعی و $M$ یک $-R$مدول مدرج با تحلیل آزاد مدرج به صورت
\begin{displaymath}
\xymatrix{
{\bf F}:~~{\cdots} \ar[r] & {F_i}\ar[r]^{d_{i}}&{F_{i-1}} \ar[r]^{d_{i-1}}  &{\cdots} \ar[r]^{d_2} & F_1 \ar[r]^{d_1} &F_0 \ar[r]^{d_0} &M \ar[r] &0}
\end{displaymath}
باشد.گوییم تحلیل فوق مینیمال است هرگاه برای هر
$i\geq 1$،
$\Ima (d_i)\subseteq {\mm} F_{i-1}$،
و به عبارت دیگر نگاشت‌های همبافت ${\bf F}\underset{R}{\otimes}R/{{\mm}}$
،همه برابر صفر باشند.
\end{definition}
%%%%%%%%%%%%%%%%%%%%%%%%%%%%%%%%%%%%%%%%%%%
                                                            %  شروع بخش دوم
%%%%%%%%%%%%%%%%%%%%%%%%%%%%%%%%%%%%%%%%%%%
\section{{عدد نظم کاستلنیوو-مامفورد}}\label{sec mumford}
در این بخش به معرفی عدد نظم کاستلنیوو-مامفورد  به همراه برخی تعاریف  معادل با آن و یک سری قضایا و نتایج در مورد آن می‌پردازیم.\\
فرض بر این است که $S=K[x_1,\ldots,x_n]$ حلقه چندجمله‌ای‌ با ساختار استاندارد مدرج روی میدان $K$ با ایده‌ال ماکسیمال همگن ${\mm}=(x_1,\ldots,x_n)$ است.
\begin{definition}[\large عدد نظم کاستلنیوو-مامفورد]\label{reg-1}
فرض کنید $M$ یک $-S$مدول مدرج متناهی مولد  با تحلیل آزاد مینیمال مدرج به صورت 
\begin{displaymath}
\xymatrix{
\displaystyle 0\ar[r] &F_s\ar[r]&\cdots\ar[r] &F_i\ar[r]&{\cdots} \ar[r] &F_0\ar[r] &M\ar[r]&0}
\end{displaymath}
باشد.(توجه شود که بنابر قضیه سیزیجی هیلبرت 
\ref{syz}
و گزاره 
\ref{pd-free}
طول این تحلیل متناهی و با بعد پروژکتیو $M$ برابر است.).
\\
گیریم $b_i$ ماکسیمم درجه از مولدهای $F_i$  باشد.برای عدد صحیح $ًr$، مدول $M$ را $-r$منظم گوییم هرگاه برای هر $i=0,\ldots,s$، 
$b_i-i \leq r$ 
باشد.
و عدد نظم کاستلنیوو-مامفورد یا به اختصار عدد نظم $M$ را به صورت زیر تعریف می‌کنیم
$$\reg(M)=\min\big \{r:~b_i-i\leq r ~~ ; i=0,\ldots,s\big\}$$
که این خود معادل است با
$$\reg(M)=\max\big \{b_i-i  :~ i=0,\ldots,s\big\}.$$
بنابراین، می‌توان گفت که $S$-مدول $M$، 
$r$-منظم
 است هرگاه $\reg(M)\leq r$.
بعلاوه  اگر $M=0$ تعریف می‌کنیم $\reg(M)=-\infty$.
\end{definition}
%%%%%%%%%%%%%%%%%%%%%%%%%%%%%%%%%%%%%%%%%%%
                                                            %  شروع بخش سوم
%%%%%%%%%%%%%%%%%%%%%%%%%%%%%%%%%%%%%%%%%%%
